\documentclass[]{scrartcl}
\usepackage[noheader, noheadertitle]{cleancv}
\usepackage{indentfirst}

\newcommand{\CVname }{Travis Aaron Hoppe}

\titleformat{\section}[block]{\noindent\bfseries}{}{0em}{} 
\titleformat{\subsection}[block]{\noindent\itshape}{}{0em}{} 

\setlength\parindent{2ex}
\setlength{\marginparwidth}{3.5cm}

\begin{document}
\begin{cleanCV}

\hspace{-2cm}
\begin{minipage}{14 cm}
\drawcvheadertitle{2cm}
\setlength\parindent{2ex}
\setlength{\marginparwidth}{3.5cm}

\LargeTitle{Teaching Philosophy}
\vspace{1em}

\textit{With a combination of didactic and Socratic seminars my teaching philosophy has always been the same; 
the critical step in teaching is letting students build the foundations themselves.}
\vspace{1em}

Knowledge in this world is not clean.
It is hard fought; a rough scrabble of empirical facts clustered into models and theories. 
The models themselves are tenuous, lasting until countered by new empirical evidence.
If knowledge were to solely subsist of facts, teaching might be considered a disingenuous trade.
However, it is the iteratively refined theories and hypotheses themselves that form the bedrock of our intellectual foundations.

The physical sciences have arguably undergone some of their most radical changes over the last century.
The clockwork mechanics of the heavens, once considered true, have been replaced by the unsettling indeterminacy of quantum mechanics along with the warped perceptions of general relativity. 
The frustratingly mesoscopic scale of molecular biology confounds a tidy theoretical description despite singular advances.
On the scale that humans can observe, the world is clearly not what it seems. 
Yet we continue to teach the current theories knowing that they may be supplanted in the future.
It is not a failure of the pedagogy, nor is it due to inadequacies of those teaching. 
The answer is more subtle and presents profound insight to our understanding and mastery of the world.

We learn by building. 
At a young age we build categories, partitions into which we place our observations.
These blocks are built haphazardly, overlapping with no thought to overall structure. 
The process cannot continue indefinitely as we have a finite cognitive capacity. 
Our brains are finely adapted to build up, tear down and structure data until it can be explained in the least number of categories.
 Often at the end, what is left is depth rather than breadth. 
Categories are built from each other, assembling complex concepts from simpler components. 
We see the same strength in a concise mathematical formula or a profound literary quote. 
The statements that encapsulate a broad collection of observations form a larger more important building set. 
This set is the basis of our collective observation.
Knowledge is the scaffolding from which we hang these observations.

This presents a unique opportunity for me as a teacher. 
I see two choices.
The first method amounts to smashing and reassembling their structure as we see fit.
While superficially effective, it misses the larger point of synthesis.
Despite the fact that the students learn to categorize observations and place them into the correct bin of knowledge, they lack the experience of building and razing the structure itself. 
In this way, knowledge itself becomes static. 
The alternative method is simple: build smaller examples for the students to deconstruct themselves.
This allows students to learn the subject in a way that builds the foundation first and supports retention through internalization.

For the last decade, I've mentored and taught students in under multiple roles.
I began formally as a teaching assistant and lab instructor and have supervised many courses on an instructor level, designed curricula, and developed new programs within the physics department.
The twenty-two courses I supported span a wide variety of physical subject matter, student composition, and skill sets.
My passion for teaching was often reflected through formal recognition\footnote{Awarded Teaching Assistant of the Year (2007) and nominated twice.}.
Throughout this time I've always focused on letting students discover and build the knowledge for themselves.



\end{minipage}

\end{cleanCV}
\end{document}
