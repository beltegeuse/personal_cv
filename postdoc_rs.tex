
\NewWorkExperience{}{{\bfseries Research Statement}}{}

\noindent\spacedlowsmallcaps{Introduction} \vspace{.5em}

\Description{The scope and accessibility of science has grown tremendously over the years. With that, the individual topics have grown into  collaborative multi-disciplined fields. The field of biophysics is no exception and has attracted biologists, chemists, physicists and applied mathematicians. What was previously studied only under a microscope, is now subjected to the exotic and unique tools that each scientist brings to the table. Lasers, x-ray crystallography, and stochastic differential equations are but a few examples. It is no surprise that biophysics is one of the hottest and rapidly growing of the multi-discipline ventures as it deals with the core constitutes of life on Earth. My research, done in collaboration with Dr. Jian-Min Yuan in the Physics department at Drexel, attempts to model one of the basic building blocks, the protein.
}

\noindent\spacedlowsmallcaps{Relevance} \vspace{.5em}

\Description{Proteins are without a doubt, an essential component to biological life. They participate in almost every process within an organism ranging from cell signaling, immune responses, to enzymes reactions. From a na\"{\i}ve perspective, a protein is simple. It is the encoding of $20$ amino acids onto a backbone - much like how the alphabet of DNA forms its words with it's four base pairs. In it's native environment however, a protein folds into marvelously complex shapes. It is often the case that the final shape of the protein ultimately determines the function. It is known that a number of diseases are associated with pathological protein folding. In the case of Alzheimer's disease an incorrectly folded protein will self assemble and aggregate into a neurotoxic species, an unexpected and highly undesirable result!}

\noindent\spacedlowsmallcaps{Research Description} \vspace{.5em}

\Description{It has long been known under what \emph{conditions} specific proteins misfold, however our theoretical understanding of the process has sorely been lacking. In a rough sense, a biologist attempts to classify, a chemist studies rates, and a physicist looks at mechanics, or the core principles governing the system. While the basic forces underlying protein folding are understood, the number of atoms involved in anything but a trivial calculation renders a closed solution, or even a full computational simulation all but impossible. The science then is reducing the complexity of the system down to a set of core ideas - from which theoretical predictions can be made. The model that we have developed over the years is a statistical one to calculate the density of states directly using few approximations of the real system.
Far from being merely physics jargon, the density of states is an incredibly useful tool to understand the underlying nature of any system. The idea is as simple as counting the number of times some particular event \emph{could} happen. As an example, if we knew to  the density of states of ages over the US population (ie. how many in the 18-25 bracket, 25-30 bracket etc...) and the propensity for each population to buy a rock album then we could accurately estimate the sales of the next big thing in music. Likewise for our biological proteins, by calculating the density of states we can predict folding (and misfolding) rates under specific conditions. Our research has been able to investigate protein folding under two very important physical constraints, crowded environments and aggregated species. The results of our research will hopefully guide experimentalists to design and predict not only these conditions within their known values, but to have a model in which \emph{a prori} unknown knowledge of the system can be formulated.}

\pagebreak

\noindent\spacedlowsmallcaps{Qualities of a Scientist} \vspace{.5em}

\Description{Developing a model and following the scientific method is only the starting point for science. My years of teaching at Drexel have taught me that science \emph{requires communication}. Without the skill to educate your audience the results become meaningless (or worse, irrelevant). This has always been true when communicating to the public, but has become especially important when different fields of science overlap and collaborate. I've worked especially hard on the communication aspect to the point of developing a specific tool for this very purpose. {\bfseries Lit-py} is a python-based literate programing paradigm that I and others have used to comment code and increase accessibility to others. When I graduate, my goal is not leave with simply with a Ph.D., an accomplishment that recognizes an outstanding volume of \emph{personal} scientific work, but know that my ideas and those of my field, have been advanced for the good of the community at large.}

\enlargethispage{\baselineskip}
\end{cleancv}

\begin{comment}
\newpage

\begin{cv}{
\spacedallcaps{Travis Aaron Hoppe}}
\vspace{1.0em}

\noindent{\bfseries \textsc Statement of Research Goals} \vspace{.5em}

My academic training started at the University of Nevada, Reno where I obtained a Bachelor of Science in both Physics and Mathematics. I continued my studies for a Masters Degree in Physics at Drexel University with a primarily focus on the kinetics and modeling of the signal transduction pathways of MAPK. My research continued at Drexel and I ultimately obtained my Ph.D. Degree in Physics studying the role of entropy in the protein folding process.

My training and expertise of biological systems is the combination of computational studies with models that have been analytical derived from core principles of the field. In particular, I have worked extensively with Monte Carlo algorithms in an attempt to study conformational states of the protein folding process, often in the presence of crowders. My primary scientific objective is to explore the crowding phenomena and develop the basic models to incorporate more physically relevant characteristics. Since derivations and modeling constitute only half of the scientific equation I anticipate that I will undergo training in the experimental aspects associated with my work. In addition, I aim to improve my ability to communicate with researchers in diverse fields far from my original training.

I believe that the laboratory of \textit{Dr. Minton} in the \textit{Physical Biochemistry Branch} (NIDDK) is an excellent scientific environment that will enable me to expand and enrich my knowledge of complex biological systems, both from the basic science and methodology perspectives. Through my training I anticipate acquiring the skills and the scientific abilities that will enable me to develop into an independent investigator.
\end{cv}
\end{comment}
