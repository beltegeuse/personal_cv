\documentclass[]{scrartcl}
\usepackage[noheader]{src/cleancv}
\usepackage{indentfirst}
\usepackage{graphicx}
\usepackage{siunitx}

\usepackage{setspace}
\renewcommand{\baselinestretch}{1.30} 

\newcommand{\CVname}{Travis Aaron Hoppe}

\renewcommand{\thesection}{}
\renewcommand{\thesubsection}{}

\titleformat{\section}[block]{\noindent\bfseries}{}{0em}{} 
\titleformat{\subsection}[block]{\noindent\itshape}{}{0em}{} 
\setlength\parindent{2ex}
\setlength{\marginparwidth}{3.5cm}

\titlespacing*{\section}
{0pt}{1ex}{1ex}
\titlespacing*{\subsection}
{0pt}{1ex}{1ex}

\newcommand{\insetpicture}[1]{\marginpar{\includegraphics[width=3.5cm]{#1}}}

\begin{document}

\begin{cleanCV}
\LargeTitle{Research Statement: Executive Summary}
\vspace{2em}

My research plan involves three primary topics. 
The first involves the development of improved implicit solvation models for protein-protein interactions focusing on the anisotropic nature of the electrostatic interaction.
The goal of this project would be the prediction of both structural proprieties like the radial distribution function and thermodynamic equilibrium properties like phase separations. 
Ideally, these models will be a function of ionic strength, pH and protein concentration.
The second project involves an application of graph theory to the structural proprieties of macromolecules.
The connection between the geometrical structure and the topological contact map, represented as a graph, will give insight along biologically relevant reaction coordinates.
The final project is a proposal to develop new enhanced sampling methods for use in simulation.
In addition to algorithmic development, I am proposing the creation of a unified set of benchmarks for all sampling algorithms to quantify the strength and applicability to a common core of problems.
\end{cleanCV}

\end{document}
