\documentclass[]{scrartcl}
\usepackage{src/cleancv}

\newcommand{\CVtype}{Curriculum Vit\ae}
\newcommand{\CVname}{Travis Aaron Hoppe}
\newcommand{\CVemail}{travis.hoppe@gmail.com}
\newcommand{\CVdegrees}{PhD Physics}
\newcommand{\CVphone}{$(775) \ 287\hbox{-}4033$}

\titleformat{\section}[block]{\noindent\bfseries}{}{0em}{} 
\titleformat{\subsection}[block]{\noindent\itshape}{}{0em}{} 

\titlespacing*{\section}
{0pt}{2ex}{1ex}
\titlespacing*{\subsection}
{0pt}{1ex}{1ex}

\begin{document}
\begin{cleanCV}

\SectionHead{Education}

\WorkExperience
{2014-current}
{Postdoctoral}
{
Theoretical Biophysics with Robert Best at the
National Institute of Diabetes and Digestive and Kidney Diseases (NIDDK), 
part of the National Institutes of Health (NIH).
}

\WorkExperience
{2011-2014}
{Postdoctoral}
{
Physical Biochemistry with Allen Minton at the NIH, NIDDK.
}


\WorkExperience
{2011}
{Doctor of Philosophy}
{
Physics, Drexel University with Jian-Min Yuan. 
\href{https://idea.library.drexel.edu/islandora/object/idea:3488}{Thesis},
\emph{On the Role of Entropy in the Protein Folding Process}.
}

\WorkExperience
{2008}
{Master of Science}
{Physics, Drexel University}

\WorkExperience
{2005}
{Bachelor of Science}
{Physics, University of Nevada}

\WorkExperience
{2005}
{Bachelor of Science}
{Mathematics, University of Nevada}


%\SectionHead{Experience}
%
%\WorkExperience
%{2011-current}
%{Research Associate (NIH, NIDDK)}
%{%
%  Developed multi-scale theoretical and computational models to study protein folding, structure, and protein-prote%in interaction.
%Worked in collaboration with experimentalists to test and validate models. 
%Designed large-scale parallel projects (1000+ cores) to investigate results. 
%}
%
%\WorkExperience
%{2005-2011}
%{Teaching Assistant (Drexel)}
%{%
%  Organized, taught, and ran over 22 undergraduate courses.
%  Developed lesson plans and curricula.
%  Restructured the entire computational component for physics majors by transitioning from FORTRAN to Python.
%}

\SectionHead{Publications}

\subsection{Protein-Protein Interaction}

\Paper
{2016}
{Incorporation of Hard and Soft Protein-Protein Interactions into Models for Crowding Effects in Binary and Ternary Protein Mixtures}
{Travis Hoppe \& Allen Minton}
{http://pubs.acs.org/doi/abs/10.1021/acs.jpcb.6b07736}
{Journal of the Physical Chemistry B}

\Paper
{2015}
{Dependence of Internal Friction on Folding Mechanism}
{Wenwei Zheng, David De Sancho, Travis Hoppe \& Robert B. Best}
{http://pubs.acs.org/doi/abs/10.1021/ja511609u}
{Journal of the American Chemical Society}

\Paper
{2015}
{An equilibrium model for the combined effect of macromolecular crowding and surface adsorption on the formation of linear protein fibrils}
{Travis Hoppe, Allen Minton}
{http://www.sciencedirect.com/science/article/pii/S0006349514048115}
{Biophysical Journal}

\Paper
{2013}
{A simplified representation of anisotropic charge distributions in proteins}
{Travis Hoppe}
{http://link.aip.org/link/doi/10.1063/1.4803099}
{Journal of Chemical Physics}

\Paper
{2013}
{Singular Value Decomposition of the Radial Distribution Function 
for Hard Sphere and Square Well Potentials}
{Travis Hoppe}
{http://dx.doi.org/10.1371/journal.pone.0075792}
{PLoS ONE}

%\Paper
%{2011}
%{On the Role of Entropy in the Protein Folding Process}
%{Travis Hoppe}
%{http://idea.library.drexel.edu/handle/1860/3488}
%{Doctoral Thesis}

\Paper
{2010}
{Protein Folding with Implicit Crowders: 
  A Study of Conformational States Using the Wang-Landau Method}
{Travis Hoppe, Jian-Min Yuan}
{http://pubs.acs.org/doi/abs/10.1021/jp107809r}
{Journal of Physical Chemistry B}

%\pagebreak

\subsection{Protein Topology \& Graph theory}

\Paper
{2014}
{Integer sequence discovery from small graphs}
{Travis Hoppe, Anna Petrone}
{http://www.sciencedirect.com/science/article/pii/S0166218X15003704}
{Discrete Applied Mathematics}

\Paper
{2009}
{Entropic flows, crowding effects, and stability of asymmetric proteins}
{Travis Hoppe, Jian-Min Yuan}
{http://link.aps.org/doi/10.1103/PhysRevE.80.011404}
{Physical Review E}

\pagebreak

\subsection{Experimental Modeling}

\Paper
{2014}
{Programmable Nanoscaffolds that Control Ligand Display to a G-Protein Coupled-Receptor in Membranes allow Dissection of Multivalent Effects}
{Andrew Dix, Daniel Appella, Travis Hoppe, \etal}
{https://pubs.acs.org/doi/abs/10.1021/ja504288s}
{Journal of the American Chemical Society}

\Paper
{2014}
{Quantification of plasma HIV RNA using chemically engineered peptide nucleic acids}
{Chao Zhao, Daniel Appella, Travis Hoppe, \etal}
{http://www.nature.com/ncomms/2014/141006/ncomms6079/abs/ncomms6079.html}
{Nature Communications}

\Paper
{2008}
{The importance of EBIT data for Z-pinch plasma diagnostics}
{A S Safronova, Travis Hoppe, \etal}
{http://dx.doi.org/10.1139/P07-170}
{Canadian Journal of Physics}

\Paper
{2006}
{Spectroscopic and Imaging Study of Combined W and Mo-pinches 
  at 1 MA-pinch Generators}
{Alla Safronova, Travis Hoppe, \etal}
{http://dx.doi.org/10.1109/TPS.2006.878361}
{IEEE Transactions on Plasma Science}


\SectionHead{Research in Progress}

\PaperX
{Enhancing the coevolutionary signal}
{Travis Hoppe, Robert Best}

\PaperX
{Stabilizing membrane proteins from thermophilic evolution}
{Travis Hoppe, Robert Best}

\PaperX
{Enchanting the coevolutionary signal}
{Travis Hoppe, Robert Best}

\PaperX
{Effect of Charge Anisotropy and the Theoretical Prediction of Phase Separation of Ionic Solutions}
{Travis Hoppe, Allen Minton}

\PaperX
{Rise of the Coauthor: Editorial on the systematic trends of coauthorship}
{Travis Hoppe, Daniel Appella}

\PaperX
{Entropic microscopes of PNA on membranes}
{Travis Hoppe, Daniel Appella}

\PaperX
{Dual-graph representation of RNA structure}
{Travis Hoppe, Tamar Schlick}

%\PaperX
%{Operator Expansion of the Grand Partition Function of Potts Models}
%{Travis Hoppe, Jian-Min Yuan}

%\PaperX
%{Evolutionary protein charge anisotropy}
%{Travis Hoppe, Jenny Hinshaw}

\SectionHead{Conferences}

\WorkExperienceX
{2016}
{Biophysical Society: Los Angeles}
{Poster: Coevolutionary signal enhancement}

\WorkExperienceX
{2015}
{Biophysical Society: Baltimore}
{Seminar: Mean-field lattice-model IDPs, Binding Affinity \& Specificity}

\WorkExperienceX
{2014}
{Advances in Enhanced Sampling Algorithms: Telluride}
{Seminar: Topological considerations in the Wang-Landau algorithm}

\WorkExperienceX
{2013}
{Biophysical Society: Philadelphia}
{Seminar: Coarse-grained Electrostatic Models for Protein Solutions}

\WorkExperienceX
{2010}
{Biophysical Society: San Francisco}
{Poster: Wang-Landau Density of States in Crowded Protein Environments}

\WorkExperienceX
{2009}
{Drexel University Libraries' Communication Symposium:\\The Hidden Costs of Scholarly Communication}
{Invited Panel Member}

\WorkExperienceX
{2009}
{Biophysical Society: Boston}
{Poster: Exhaustive Properties of Simple Lattice Peptides}

\pagebreak


%\pagebreak

\SectionHead{Teaching (Drexel)}

\newcommand{\TeachingNote}{$^*$}

\Teaching
{2011}
{PHYS 305}{Computational Physics II\TeachingNote}

\Teaching
{2010}
{PHYS 304}{Computational Physics I\TeachingNote}

\Teaching
{}
{PHYS 160}{Introduction to Scientific Computing\TeachingNote}

\Teaching
{}
{PHYS 305}{Computational Physics II\TeachingNote}

\Teaching
{2009}
{PHYS 304}{Computational Physics I\TeachingNote}

\Teaching
{}
{PHYS 160}{Introduction to Scientific Computing\TeachingNote}

\Teaching
{}
{DSP 099}{Dragon Summer Program: Remedial Mathematics}

\Teaching
{}
{PHYS 100}{Preparation for Engineering Studies}


\Teaching
{}
{PHYS 305}{Computational Physics II\TeachingNote}

\Teaching
{2008}
{PHYS 304}{Computational Physics I\TeachingNote}

\Teaching
{}
{PHYS 102}{Fundamentals of Physics II\TeachingNote}

\Teaching
{}
{PHYS 115}{Contemporary Physics III\TeachingNote}

\Teaching
{}
{PHYS 114}{Contemporary Physics II\TeachingNote}

\Teaching
{2007}
{PHYS 113}{Contemporary Physics I\TeachingNote}

\Teaching
{}
{PHYS 102}{Fundamentals of Physics II, Lab}

\Teaching
{}
{PHYS 115}{Contemporary Physics III\TeachingNote}
\Teaching
{}
{PHYS 114}{Contemporary Physics II\TeachingNote}

\Teaching
{2006}
{PHYS 113}{Contemporary Physics I\TeachingNote}

\Teaching
{}
{TDEC 101}{Fundamentals of Physics I, Lab}

\Teaching
{}
{TDEC 103}{Fundamentals of Physics III\TeachingNote}

\Teaching
{}
{TDEC 102}{Fundamentals of Physics II\TeachingNote}

\Teaching
{2005}
{TDEC 101}{Fundamentals of Physics I\TeachingNote}

\TeachingNote{\small Developed new curricula and modernized the Computational Physics, Contemporary Physics and Introduction to Scientific Computing courses at Drexel.}


\SectionHead{Awards}

\WorkExperienceX
{2014}
{Top Presentation Award}
{Institution-wide recognition during the NIDDK Annual Conference.}

\WorkExperienceX
{2010}
{Research Assistant Grant}
{Competitive grant from Drexel Physics Department on the basis of outstanding research and teaching.}

\WorkExperienceX
{2010}
{Student Research Achievement Award (SRAA)}
{Top poster at the Biophysical Society 2010 meeting. }

\WorkExperienceX
{2009}
{Department Research Award (Senior Division)}
{Given by the Drexel Physics Department, this award recognized a high proficiency in both original research and synthesis of results into publications.}

\WorkExperienceX
{2008}
{Department Research Award (Junior Division)}
{Restricted to the first two years of study, the junior division award was awarded for early achievements in research.}

\WorkExperienceX
{2007}
{Teaching Assistant of the Year} {}
%{Recognition by Drexel University as the top Teaching Assistant in the College of Arts and Sciences.}

%\SectionHead{Open-source Projects}
%\WorkExperienceX
%{2009}
%{Literate Python}
%{}
%\JobDesc
%{lit-py: }
%{An open source 
%\href{https://bitbucket.org/hooked/lit-python/overview}
%{pedagogical tool}
%to document code.
%Equations and code output are seamlessly integrated into a \LaTeX\ document and complied into a PDF for easy distribution. 
%Used by members of the Drexel physics department to both teach and share code among students and peers.}

\end{cleanCV}
\end{document}
